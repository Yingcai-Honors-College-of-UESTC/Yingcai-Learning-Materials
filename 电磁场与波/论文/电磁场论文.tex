\documentclass[a4paper]{article}
\usepackage[margin=1in]{geometry}%设置边距,符合Word设定
\usepackage{amssymb,amsfonts,amsmath,amsthm}
\usepackage{ctex}
\usepackage{setspace}
\usepackage{lipsum}
\usepackage{graphicx}%插入图片
\graphicspath{{Figures/}}%文章所用图片在当前目录下的 Figures目录

\usepackage{hyperref} % 对目录生成链接,注:该宏包可能与其他宏包冲突,故放在所有引用的宏包之后
\hypersetup{colorlinks = true,  % 将链接文字带颜色
	bookmarksopen = true, % 展开书签
	bookmarksnumbered = true, % 书签带章节编号
	pdftitle = 学习电磁规律,增长思维能力, % 标题
	pdfauthor =刘正浩 2019270103005} % 作者
\bibliographystyle{plain}% 参考文献引用格式
\newcommand{\upcite}[1]{\textsuperscript{\cite{#1}}}

\renewcommand{\contentsname}{\centerline{Contents}} %经过设置word格式后,将目录标题居中


\title{\heiti\zihao{2} 学习电磁规律,增长思维能力}
\author{\songti 刘正浩 2019270103005}
\date{\today}


\begin{document}
	\maketitle
	\thispagestyle{empty}


	\tableofcontents

	\section{课程对我的启发}
		在学习电磁场与波这门课程之前,我对电磁场的认识是片面的、不充分的。由于高中学习过静电场与静磁场的简单分析,而没有学过时变场的相关内容,
		导致我对“电磁场”这个词的印象仅仅局限于时变场相关的知识,没有考虑到静态场与恒定场在电磁场理论中的地位。\par
		在学习了电磁场与波课程之后,我对电磁场的了解更加深入了。虽然大学物理课程中也有电磁场的相关内容,但更多的还是对高中物理的延伸和补充,
		也就是更加详细地讲解了有关静态场的知识;至于时变场的内容则并未过多的涉及,只粗略地介绍了麦克斯韦方程组和有关的一些内容。\par
		通过学习电磁场与波这门课程,我能将一些生活中常见的电磁现象与电磁场的具体知识连接起来,例如开关电源产生的高频杂波对其附近的电路的影响,
		以及不同封装的电容在容值方面的差异,等等。\par
		课程中对于电磁场理论的发展过程的讲解很详细,例如,老师详细讲解了麦克斯韦方程组的发展过程,这样的讲解使我对电磁场理论的发展过程更加了解,
		这也有助于我对电磁场这门课的学习与理解。
	\section{对“立足根本,化繁为简”科学研究方法的认识和体会}
		在课程中充满了“立足根本,化繁为简”的科学研究方法。这其中给我印象最深的是求解静电场的相关部分。\par
		在高中甚至大学物理的学习过程中,对静电场的求解始终是一个重点、难点。高中物理基本上只涉及了均匀电场、均匀磁场的分析以及简单电磁感应现象的分析。
		这些分析基本上用代数方程就可以解决。大学物理相比高中物理,难度上了一个台阶。大学物理主要分析有源场的场源关系以及一些简单的电磁波分析,其中一部分问题可以用代数方程解决,
		另一部分需要用微积分知识进行求解,但涉及到的相关公式简单,因此分析过程也比较简单。而电磁场课程中基本上涉及了所有类型电磁场的规律、分析,几乎所有的问题都依赖微积分知识求解。
		不过这其中有很大一部分内容可以进行简化。\par
		在对一个场进行分析时,首先要分辨这个问题是一个一维问题还是多维问题。对于一维问题,基本上我们都可以通过建立特殊的坐标系来建立更简单的方程进行求解,
		这个方程可以是代数方程,也可以是微积分方程,然后将微积分方程通过一定的公式简化为代数方程进行求解。对于高维、复杂问题,则可以通过镜像法、分离变量法、有限差分法等方法进行简化,
		将这些高维、复杂的问题转化为一维、简单的问题,之后就可以应用求解一维问题的方法进行求解。这就是“化繁为简”的一个典型例子。\par
		此外,对于同一个问题有许多不同的求解方法,这也是电磁场这门学科的魅力所在。举个例子,对于静电场的求解,利用同样的条件,既可以用叠加原理进行求解,还可以用高斯定理进行求解,
		此外还有电位泊松方程、亥姆霍兹定理等等不同的方法,都可以用来求解同一个问题。
	\section{第四版教材与讲义的差异性感受}
		在我看来,第四版教材与讲义的最大差异在于,教材上对于一些基本概念并没有详细说明,而是隐藏在正文中,这就导致在自己看书时可能会忽略一些概念。而在讲义中则明确地标出了一些基本概念和重要公式,
		方便进行自习和查询。\par
		此外,教材对某些知识点的讲解过于简单,对另一些知识点的讲解又过于详细。这就导致我在看书时有些迷茫:一方面有一些问题无法从书上获得足够的信息,只能去网上搜索相关资料;
		另一方面,有一些简单的问题书上却花了较大的篇幅进行介绍,有些多余。而讲义则对所有的知识点都有涉及,只不过在知识点的详细度上来说讲义与教材还有一定差距,有一些计算上的问题在讲义中没有体现,
		导致在自习中存在一些困难。\par
		在例题方面,我在看教材时习惯于通过分析例题来对某一个知识点进行更深入的理解,而讲义中例题相较于教材中的例题更少,所以我在自习时经常会陷入这样一种状态:在看知识点讲解时是通过看讲义来解决,
		在巩固知识点时通常使用教材上的例题,而教材与讲义的知识结构又不太一样,所以会在教材和讲义之间翻来翻去,也会占用一些时间。\par
		此外我还有一个小建议,就是教材与讲义在变量的含义上并没有进行详细说明。而在计算中需要用到许多十分相近的变量,例如在球坐标系中的$\rho$和表示体密度的$\rho$,当它们出现在同一个公式中时,
		就需要猜它们各自表达什么物理量,这也会耗费一些时间。
	\section{课程教学的建议与评价}
		在上课时潘老师更多地讲解了电磁场与波这门学科的基本思想和核心内容、核心方法,这对于我们掌握电磁场的最基本的概念有很大帮助。但是由于这门课程考试的一部分难点是在具体问题的分析与计算上,
		所以我希望课程中能穿插几节习题课,让我们对具体的解题方法也有一定的了解。\par
		在课程中老师主要对电场进行了分析,而对磁场,尤其是磁矢位的分析有些偏少,在看书以及做课后习题的课程中我依然对磁矢位不够熟悉,希望讲义中可以多一些对磁场和磁矢位的分析。
		此外,我感觉时变场有些抽象,还是不太明白电场和磁场是如何耦合在一起的,如果能有一些易懂的图片或者文字描述可能更能帮助我理解时变场究竟长什么样子。

\end{document}